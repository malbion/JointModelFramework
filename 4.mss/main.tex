% DOCUMENT CLASS
    % Change "letterpaper" to "a4" if you use a4 paper size
    \documentclass[a4,12pt]{article}
    % Thesis formatting:  line spacing 1.5, Times New Roman or Arial 12 pt font, all four margins 20mm, A4 size.
	
  %  \usepackage{titlesec} % Allows customization of titles
    \usepackage{authblk} % For multiple authors
    \usepackage{amsmath}
    \usepackage[utf8]{inputenc}
    %\usepackage[T1]{fontenc}
%     \renewcommand{\rmdefault}{phv} % Arial
% \renewcommand{\sfdefault}{phv} 
 	\usepackage{times}
 	\usepackage{setspace} % lines spacing
    \usepackage{lineno} 	% See \linenumbers at the top of content.tex
    \usepackage{booktabs}
    \usepackage[skip=2.5\baselineskip]{caption} % to give more space between figs and captions
    \usepackage{listings} % allows me to insert code 
    \usepackage{pdfpages}
    \usepackage[margin=20mm]{geometry} % set exact margins for thesis

% REFERENCES
    \usepackage[firstinits=true, backend=biber, style=authoryear]{biblatex}
    \DeclareNameAlias{sortname}{last-first}

    \addbibresource{../../../BibTex_files/03_Chapter3.bib}

% GRAPHICS
    \usepackage{graphicx} % More advanced figure inclusion
    \usepackage{float} % For specifying table/figure locations, i.e. [ht!]
    \usepackage{changepage}
    \usepackage[table,xcdraw]{xcolor}

\onehalfspacing % line spacing 1.5 for thesis formatting

\title{\large Estimating interaction matrices for diverse, horizontal systems}



\author[1]{\small Malyon D. Bimler *}
\author[2]{\small Margaret M. Mayfield}
\author[3]{\small Trace E. Martyn}
\author[4]{\small Daniel B. Stouffer}

\affil[1]{\footnotesize School of Biological Sciences, The University of Queensland, St Lucia, Queensland, Australia. Email: malyonbimler@gmail.com}
\affil[2]{\footnotesize School of Biological Sciences, The University of Queensland, St Lucia, Queensland, Australia. Email: m.mayfield@uq.edu.au }
\affil[3]{\footnotesize School of Natural Resources and the Environment, The University of Arizona, Tucson, USA. Email: tmartyn@arizona.edu }
\affil[4]{\footnotesize Centre for Integrative Ecology, School of Biological Sciences, University of Canterbury, Christchurch, New Zealand. Email: daniel.stouffer@canterbury.ac.nz }



\setlength{\intextsep}{10ex}

\begin{document}
\maketitle  

\noindent
\textbf{Running title:} Estimating horizontal interaction matrices

\noindent
\textbf{Corresponding Author:} Malyon D. Bimler, email: malyonbimler@gmail.com, ph: +64 481941634\\


\noindent
Manuscript submitted for consideration as a \textbf{Method} article.


\noindent
Number of words - Abstract: 147\\
Number of words - main text: 
Figures \&  Table in main text.\\
Number of references: XX\\

\noindent
\textbf{Keywords:} Species interactions, competition, facilitation, networks, population dynamics, non-trophic, ....  

\section*{Author contributions}

M.D.B. designed the methodology, carried out analyses and led the drafting of the manuscript. D.B.S. helped design the methodology and interpret analyses and critically revised the manuscript. T.E.M. led the field study and data collection. M.M.M. helped design the field study, collected data, contributed to the interpretation of analyses and critically revised the manuscript. [DOES THIS NEED TO CHANGE??]

\section*{Data accessibility}

Data will be submitted to Dryad upon publication.

\newpage

% \setcounter{secnumdepth}{3} % sections are level 1

\linenumbers

% Submit to Ecology Letters as a Method

    \paragraph{}
    \textbf{Network theory allows us to understand complex systems by evaluating how their constituent elements interact with one another. Quantifying these interaction matrices from empirical data can be difficult however, because the number of potential interactions increases non-linearly as more elements are included in the system, and not all interactions may be empirically observable when some elements are rare. We present a novel modelling framework which estimates realised and unrealised interactions in diverse horizontal systems, using measures of individual performance in the absence and presence of their potential interaction partners. We apply our method to an ecological dataset of an annual wildflower community and quantify the competitive and facilitative effects of 52 neighbouring species on the viable seed production of 22 focal species. We illustrate how these results can help us better understand and manage plant communities, with potential applications of this framework extending well beyond plant community ecology.}



\section{Introduction}

    
    \paragraph{}
    In many biological systems, interactions between system elements (be these species, individuals etc.) affect performance and together determine the dynamics of the whole system. In order to understand system dynamics when multiple system elements are involved, such complex systems can be represented as networks where the elements are nodes and linked by interactions \parencite{Pimm1978}. These nodes can take on a wide array of identities, including cells, individuals, populations or species. Likewise, interactions or links can operate via many different mechanisms and have a wide range of effects on the nodes. Biological networks have been observed to typically differ from randomly-assembled networks in important and varied ways, thus informing us on the biological processes structuring these systems \parencite{Dunne2002, Kinlock2019}.

    \paragraph{}
    Network theory has been widely applied to investigate the structure of biological systems such as food webs and other types of multi-level ecological interaction networks. There is now a rich body of work describing the structural properties of food webs, plant-pollinator networks, and host-parasite interactions (e.g. \cite{Lafferty2008, Thompson2012, Dunne2013, Stouffer2014, Cirtwill2015a}). Horizontal networks however, where interactions occur within the same level of organisation (for example interactions between plants belonging to the same food web) have been more neglected by network ecology \parencite{Ellison2019}. These elements and their interactions have been found to alter the structure, stability and diversity-functioning relationships of the whole multi-level system \parencite{Hammill2015, Giling2019, Zhao2019, Miele2019}, which makes integrating horizontal and vertical networks a key challenge in improving our understanding of system dynamics and persistence \parencite{Godoy2018c}.

    %\paragraph{} % I reckon this paragraph can be cut?? It doesn't bring that much to the intro... Can be kept for MEE
    %In many horizontal systems, interactions between nodes are not always easy to observe empirically and must instead be deduced through other means. One strategy is to look at associations between the nodes: nodes which are often found together are assumed to interact more strongly than nodes which rarely co-occur (e.g. \cite{Araujo2011}). This approach benefits from a wide range of packages available in R, requires fairly simple data (presence/absence of the nodes) and can be used to deduce associations between large numbers of nodes. An important drawback remains: association networks cannot always distinguish between the multiple confounding factors which lead to co-occurrence, and in ecological communities little evidence supports their use as a proxy for interactions \parencite{Sander2017,Barner2018, Thurman2019, Blanchet2020}.
    %% [Keep this if I have enough spare words]

    \paragraph{} 
    In many horizontal systems, interactions between nodes are not always easy to observe empirically and must instead be deduced through other means. A common approach in ecology is to directly quantify the effects of interacting nodes on the node of interest by evaluating its performance in the absence and presence of potential interaction partners \parencite{Connell1961, Grace1990}. 'Performance' can refer to any variable of interest that affects the dynamics of the system, for example [examples? energy generated, electrical activity, nutrients exchanged,..] population growth. The resulting interactions are phenomenological and thus not dependant on any specific mechanism, allowing us to capture a wide range of biological processes affecting the dynamics of the whole system \parencite{Novak2010}. Such methods can quickly become data intensive and computationally complex, however, as the number of nodes $S$ increases and the number of potential direct interactions subsequently increases to $S^2$. Highly diverse systems pose a further challenge: the abundance distribution of different elements is typically skewed, with a few nodes making up the majority of abundances and a large number of nodes remaining rare \parencite{Fisher1943}. Given that data collection is limited in time and scope, interactions with rarer nodes especially may not be realised simply by chance, excluding them from analysis regardless of the role they might play \parencite{Olesen2011}. Empirically quantifying interaction matrices for diverse horizontal systems thus requires a method that is flexible to both a high number of nodes, and potential gaps in our records of interactions. 


    \paragraph{} 
    We present a general framework to estimate interactions in highly diverse systems using Bayesian parameter estimation methods which....... We specify a joint model which allows us to estimate both realised and unrealised interactions, the latter being conditioned on the former. Using annual wildflower community data in Western Australia consisting of 22 focal species and up to 52 neighbouring species, we illustrate how this novel framework can be applied by couching the resulting interactions into established models of population dynamics. This allows us to account for other demographic processes affecting species performance, as well as translate interaction effects on the measured variable into effects on another variable of interest. We further show the benefits of this ... approach by comparing a range of findings on our test natural system derived from a spatial - abundance association network [I don't like this!!], and suggest potential applications in species management and conservation that make use of the rich information provided by species interaction networks.

    % \paragraph{}
    % Measuring non-trophic relationships in plant communities is difficult, in part because they cannot be directly observed as in food webs or other interaction networks, and must instead be inferred by other means. Arguably the most common approach to empirically quantifying interactions between a focal species and others involves measuring the effects of neighbours on a proxy for lifetime reproductive success (e.g. fecundity, growth rate) \parencite{Connell1961, Grace1990}. This can be done for example by growing focal species with neighbours of different identities and at varying densities, in the field or laboratory. The resulting phenomenological interaction strengths are not dependant on any specific mechanism and can be directly linked to changes in species abundances through the use of population dynamic models, for example with the Beverton-Holt model \parencite{Beverton1957, Levine2009}). This approach has been widely applied and has greatly improved our understanding of competition and its effects on biodiversity \parencite{Tilman1982, Chesson2000b, Levine2008, Adler2010, Mayfield2010a, Kraft2015} but remains limited to investigating interactions between few species, rather than a diverse community. The reason is two-fold: applying this method to a large number of species requires both large amounts of data and replication, as well as complex models which can be difficult to implement and run the danger of over-parameterisation. 

\section{Methods}

[insert a paragraph from the results section here]

    \subsection{Data requirements}

    \paragraph{}
    The joint model framework was initially developed for an ecological dataset where interacting species affect each other's lifetime reproductive success. As such we refer to network elements or nodes as 'species' and node performance as 'lifetime reproductive success' or 'fitness' however this framework is not limited to ecological datasets and can be applied to data from any interacting elements (e.g. populations or groups of individuals) which meet the following criteria. First, observations must record a proxy for performance, for example seed production, biomass or stem diameter growth as proxies for lifetime reproductive success. Second, these observations must also record the identity and abundance of neighbouring elements within the interaction neighbourhood of each focal element. Lastly, observations should be replicated for each focal element with the aim to capture variation in the identities and densities of neighbouring elements which make up the different interaction neighbourhoods. [REPHRASE]

    % interaction neighbourhoods for each focal element, in both the identities and densities of the elements which constitute them. 

    In addition to these requirements, the experimental design or data may also benefit from observations of focal elements with empty interaction neighbourhoods to better estimate intrinsic performance (i.e. in the absence of neighbours). In study systems which allow it, a proportion of interaction neighbourhoods can instead be thinned prior to the experiment to randomly remove neighbouring elements and provide observations for low-density estimates of interactions. Though neither of these steps are strictly necessary, thinning certain neighbourhoods can also reduce potential confounding effects between the environment and interactions and thus provide more accurate estimates of interaction effects. The risk here is that focal elements with dense interaction neighbourhoods may only be observed when the environment is of good enough quality, or that low-density neighbourhoods are only observed when the environment is too poor to support more elements, not because
    If competition and environment quality are correlated for example, ...
    - for example ... Environmental data known to affect performance can also be recorded and included in the model (as a random effect for example) to minimise those confounding effects.

    any experimental design or data which may reduce confounding effects between environment and competition will provide more accurate estimates of interaction effects (e.g. thinning certain plots, recording environmental data known to affect performance). 
    
    \subsection{Neighbour density-dependant model}
        
        \paragraph{}
        We begin by implementing a neighbour density-dependant model (NDDM) to each focal species $i$ which regresses the identity and abundance of neighbours $j$ ($j = 1, 2, 3, ..., S$) against the measured proxy for lifetime reproductive success $P_{i}$:
        
        \begin{equation}
        P_{i} = \beta_{i0} - \sum_{j=1}^{S} \beta_{ij} N_{j}
        \label{nddm}
        \end{equation}
        
        The intercept $\beta_{i0}$ represents intrinsic fitness, a species' fitness in the absence of interactions with neighbours. $N_{j}$ are the abundances of neighbours recorded for each observation, and the $\beta_{ij}$ represent the species-specific effect of each neighbour $j$ on $i$. Note that neighbours can include conspecifics, in which case intraspecific interactions are denoted as $\beta_{ii}$.

        \paragraph{}
        This model, as well as the RIM described below, are implemented as generalised linear models. This means that the relationship between $P_i$ and the right-hand side of the equation does not need to be linear, but can be modified to fit the data in question given an appropriate distribution and link function, for example exponential or negative binomial (as was the case for our case study). 
              
    \subsection{Unrealised links}
    
        \paragraph{}
        In any given site or year, a focal species may only be observed to interact with a subset of potential interaction partners, which means the NDDM above will not be able to estimate all potential interactions ($\beta_{ij}$) between species. This is especially true for rare species. These \textit{unrealised} interactions can be estimated by using an alternative model with a different set of assumptions to the NDD model. This model is described as the response-effect model (here called the response-impact model or RIM) by Godoy, Kraft and Levine (2014) and assumes that each species has the same effect on all neighbours regardless of their identity, as well as the same response to competition regardless of competitor identity. 
        
        \begin{equation}
        P_{i} = \beta_{i0} - r_{i} \sum_{j=1}^{S} e_{j} N_{j}
        \label{rim}
        \end{equation}
        
        Pairwise interactions which are missing from Eq. 1 can be approximated by Eq. 2 by multiplying the relevant $r_{i}$ and $e_{j}$ such that $\beta_{ij} = r_{i} e_{j}$. In order to allow both the NDDM and RIM interaction estimates to contribute to the likelihood, we first used the NDDM to quantify observed interactions and then used those to estimate species-specific $r$ and $e$ parameters such that: 
    
        \begin{equation}
        r_i e_j \sim logistic \left ( \beta_{ij}, \sigma \right )
        \label{unrealised}
        \end{equation}
    
        where $\sigma$ is a community-level scale parameter for the logistic distribution. We use a logistic distribution here because the heavier tails make it a slightly weaker informative prior than a normal distribution. $\sigma$ defines how widely the tails extend and can be expressed in terms of the standard deviation. 

    \subsection{Standardising effects into per-capita interactions}

        The $\beta_{ij}$ and $r_i e_j$ estimates returned by the framework describe the effect of species $j$ on the proxy for lifetime reproductive fitness of species $i$. Differences in the magnitude of these interaction terms may thus reflect differences in reproductive fitness, which can vary widely both between species and under different environmental conditions. In order to make the effects of interactions comparable between species, the interaction terms returned by both models above can be transformed into \textit{per-capita} interaction strengths \parencite{Godoy2014, Bimler2018}. The appropriate scaling is determined by writing the neighbour density-dependant model (Eq. \ref{nddm}) into a form equivalent to a Lotka-Volterra competition model: 

        \begin{equation}
        P_{i} = \beta_{i0} \left ( 1 - \sum_{j=1}^{S} {\beta_{ij}}'' N_{j} \right )
        \label{LVform}
        \end{equation}

        This reveals that our interaction terms can be rescaled into per-capita interaction strengths by dividing them with the recipient species' intrinsic fitness:  

        \begin{equation}
        {\beta_{ij}}'' = \frac{\beta_{ij}}{\beta_{i0}}
        \label{scaling}
        \end{equation}

        where $\beta_{ij}$ can be replaced with $r_i e_j$ when appropriate. Though this scaling step is not strictly necessary, per capita interaction strengths have the benefit of being directly comparable both across species and across environmental contexts where reproductive fitness may vary \parencite{Wootton2005}, leading to a wider range of potential applications.

    \subsection{Integrating interaction strengths into models of population dynamics}

        \paragraph{}
        In certain instances, models of population dynamics can be used to further extend the usefulness of the framework presented here. We suggest two cases where such an application may be useful. Firtly, the variable chosen to measure the performance of focal individuals $P_i$ may not always directly translate into estimates of lifetime reproductive success due to inherent practical constraints with collecting empirical data. For example, certain life-history reproductive strategies may lead to measures of high seed production in the field which do not account for low seed or seedling survival rates post-observation. In these cases, population dynamics models can be used to account for species-specific demographic rates into estimates of interaction effects. Alternatively, we might be more interested in the effects of neighbours on the abundance or growth rate of a focal species rather than on it's proxy for lifetime reproductive success. In this scenario, a population dynamic model can be used to translate interaction effects on the measured variable into interaction strengths affecting the variable of interest. 

        \paragraph{}
        In both cases, an established population dynamic model is required as well as knowledge of any crucial demographic rates. This step is illustrated in our case study below, where an annual plant population dynamic model is used to transform effects on wildflower seed production (the measured proxy for lifetime reproductive success) into effects on population growth, and includes species-specific estimates of seed germination and survival rates. 

        % \paragraph{}
        % Though this rescaling is not necessary when interactions are quantified as effects on rates such as growth or biomass accumulation, it is particularly useful in cases where $P_i$ is captured by variables which are affected by r/K reproductive strategies, such as seed or fruit production. Species $i$ for example may produce a high number of seeds of which few are viable, whereas species $j$ produces very few seeds though these are much more likely to mature into reproductive adults. Scaling the interaction terms returned by the models allows us to compare how species $i$ and $j$ both respond to interactions, relative to their performance. [MOVE TO DISCUSSIO?]


    \subsection{Model fitting}

        \paragraph{}        
        The NDDM (Eq. 1) and the RIM (Eqs. 2 and 3) can be implemented as generalised linear models in STAN \parencite{Carpenter2017}, a Bayesian statistical language where coefficient values will be estimated by MCMC sampling. The advantage of this approach is two-fold: the model can converge and coefficients can be estimated despite high model complexity and a large number of parameters. Using STAN requires translating the model formula into the STAN language, setting priors for parameters to be estimated, and using an indexing system to identify \textit{realised} interactions which are then fed into the REM. We provide a working example of the STAN file used to specify and set this model to the case study below in the Supplementary Information, as well as a generalisable function in R which prepares the data and required indexing. From the model file, only the link function for $F_i$ and it's parameterisation need to be modified in order to apply it to a different model or system. Additionally, non-integer measures of performance (e.g. biomass) should be redefined as real rather than integers in the data block. In the code given, a negative binomial distribution is used to fit seed production ($F_i$) but a different distribution may be more appropriate when using other proxies for lifetime reproductive success.   

        \paragraph{}
        STAN returns parameters as distributions which maximise the likelihood, and are conditioned by the data and priors. Priors describe the distribution of plausible values which these parameters may take. For an introduction to Bayesian inference which relates the use of priors to frequentist hypothesis testing, see \textcite{Ellison1996}. We recommend investigators experiment with setting different informed priors to both improve model convergence and verify the robustness of parameter estimates. The resulting parameters are termed posterior distributions, and samples from the posterior are drawn for analysis. \textcite{Ellison2004} also provides an accessible review of parameter estimate and the use of posterior distributions using a worked example on ant species richness data.

        %----------------------------------

    \section{Case study}

        We applied the model framework described above to a case study of annual wildflower plants. After quantifying the interaction matrix, we scale all interaction samples using estimates of intrinsic fitness ($\beta_{i0}$) and experimentally-measured species demographic rates. The scaling is determined by a system-appropriate population dynamic model and allows us to translate the interaction matrix into per-capita effects on  abundance. This transformation places the resulting interaction strengths on the same scale and allows us to compare the community-wide effects of different species, amongst other potential applications further illustrated in the Results and Discussion. 
        % Maybe add justification for applying it to plant data here -  possibly add that we assume all nodes 'perform' in the same way. (ie applicable to horizontal systems)


        \subsection{Data}

        \paragraph{}
        We applied this framework to annual wildflower community dataset from Western Australia. This system is a diverse and well-studied community of annual plants which germinate, grow, set seed and die within approximately 4 months every year. Individual fecundity data were collected in 2016, when 100 50 x 50 cm plots established in the understory of West Perenjori Reserve (29$^o$28'01.3"S 116$^o$12'21.6"E) were monitored over the length of the full field season, from July to October. The resulting dataset contains from 29 to over 1000 counts of individual plant seed production from 22 different focal species (with a median of 108 observations per species), in addition to the identity and abundances of all neighbouring individuals within the interaction neighbourhood of the focal plant. Interaction neighbourhoods varied in radius from 3 to 5 cm depending on the size of the focal species. Total neighbourhood diversity was 71 wildflower species, 19 of which were recorded fewer than 10 times across the whole dataset. The species-specific effects of this latter group of species on focals were deemed negligible due to their extremely low abundance, they were thus grouped into an 'other' category and their effects on focals averaged. This resulted in 53 potential neighbour identities. Half of all plots were thinned (a quarter to 60\% diversity and a quarter to 30\%) to mitigate possible confounding effects between plot location and plant density, and did not target any particular species.

        \paragraph{} 
        We required species demographic rates (seed survival and germination) in order to scale model interaction estimates into per-capita interaction strengths. Species demographic rates for 16 of our focal species were estimated from a database of field experiments carried out between 2016 and 2019 where seedbags were placed in the field to estimate germination rates, and ungerminated seeds were evaluated in the lab for survivability.  The remaining species were assigned mean demographic rates from these experiments. See Bimler et al. (\textit{In Review}) Supplementary Methods .2 for full details on the methods used for collecting those seed rates.

        \subsection{Model fitting}

        \paragraph{}
        We fit the model using R version 3.6.3, STAN and the rstan package \parencite{R2020, Carpenter2017, Rstan2020}. Estimates of seed production were fit with a negative binomial distribution. The model was run with 4 chains of 5000 iterations each, discarding the first 1000. Models were checked for convergence and traceplots were visually inspected to verify good chain behaviour and mixing. Model parameters were sampled 1000 times from the 80\% posterior confidence intervals to construct our parameter estimates. We then applied bootstrap sampling from each resulting interaction strength distribution to create 1000 samples of the community interaction network.

        \subsection{A model for annual plant population dynamics}

        \paragraph{}
        The above model framework returns species-specific estimates of intrinsic fitness ($\beta_{i0}$), as well as as a species x neighbour matrix of observed ($\beta_{ij}$) and unobserved ($r_i e_j$) interaction estimates which quantify the effects of one neighbour $j$ on the intrinsic fitness of a focal species $i$. Though useful as they are, these estimates can lead to a wider range of potential applications when integrated into models of population dynamics. For example, we might be more interested in the effects of neighbours on the abundance or growth rate of a focal species rather than on it's proxy for lifetime reproductive success. Importantly, it is necessary to specifiy a model describing population dynamics in order to draw conclusions about the effects of interactions and network structure on the maintenance of community diversity and stability. 

        \paragraph{} 
        We defined the following model for annual plants with a seed bank \parencite{Levine2009, Mayfield2017, Bimler2018} which describe the rate of change in a focal species' \textit{i} abundance of seeds in a seed bank from one year to the next: 
    
            \begin{equation}
                \frac{N_{i, t+1}}{N_{i, t}} = \left( 1 - g_{i} \right) s_{i} + g_{i}F_{i, t}
                \label{ifm}
            \end{equation}
        
        where \(F_{i,t}\) measures the number of viable seeds produced per germinated individual whilst \(g_{i}\) and \(s_{i}\) are the seed germination and seed survival rate, respectively. In a simplified case where the focal species \textit{i} interacts with only one other species \textit{j}, in this model of population dynamics \(F_{i,t}\) is given by:

            \begin{equation}
                F_{i,t} = \lambda_{i} e^{- \alpha_{ii} g_{i} N_{i, t} -  \alpha_{ij} g_{j} N_{j, t} }
                \label{fecundity}   
            \end{equation}

        where \(\lambda_{i}\) corresponds to seed number in the absence of competition, and \(\alpha_{ii}\) and \(\alpha_{ij}\) are the interaction strengths between species \(i\) and its intraspecific and interspecific neighbours respectively. Here it is \(\alpha_{ij} g_{j}\) and \(\alpha_{ii} g_{i}\) which are equivalent to \(\beta_{ij}\) in Eq. \ref{nddm}. 
        We determine the scaled, per-capita interaction strengths ${\alpha}''$'s by including \(\lambda_{i}\), \(g_{i}\) and \(s_{i}\) in such a way that these variables are cancelled out when the ${\alpha}''$'s are substituted for the $\alpha$'s in our annual plant population model. 

        \begin{equation}
            {\alpha_{ij}}'' = \frac{g_{j} \alpha_{ij}}{ln(\eta_{i})}
        \end{equation}

        with $\eta_{i} = \frac{\lambda_{i} g_{i}}{\theta_{i}}$ and $\theta_{i} = 1 - (1 - g_{i})(s_{i})$. % \(ln(\eta_{i})\) is thus equivalent to \(\beta_{i0}\) in Eqs. \ref{nddm} and \ref{scaling}.
        Note that because our model evaluates the rate of change of seeds in the seed bank, 'per-capita' here refers to changes 'per seed in the seed bank' of a focal species. Substituting ${\alpha}''$'s for $\alpha$'s in Eq. \ref{ifm} gives us: 
        
        \begin{equation}
            \frac{N_{i, t+1}}{N_{i, t}} = (1 - \theta_{i}) + \theta_{i} \eta_{i} e^{-ln(\eta_{i})({\alpha_{ii}}'' N_{i, t} + {\alpha_{ij}}'' N_{j, t})}
        \end{equation}

        where we can see that the ${\alpha}''$'s are directly proportional to the abundance of neighbours. Relating this population model to the joint model framework, we recover the following: 

        \begin{equation}
        {\beta_{ij}}'' = {\alpha_{ij}}''
        \end{equation}

        \begin{equation}
        \beta_{ij} = g_{j} \alpha_{ij}
        \end{equation}

        \begin{equation}
        \beta_{i0}  = ln(\eta_{i}) = ln(\frac{\lambda_{i} g_{i}}{\theta_{i}})
        \end{equation}

        % \begin{equation}
        % P_{i, t} = g_i F_{i, t} = ln(g_i F_{i, t})
        % \end{equation}

        % which one is it - and how does it match the code?!

        As we show here, the exact form of the rescaled interactions as well as intrinsic fitness can therefore vary depending on the specific population dynamic model applied and may include other demographic rates which reflect species-level differences in growth and mortality. Because intrinsic fitness is estimated by the model framework and not directly observed, we used the mean of the $\lambda_{i}$ posterior distribution returned by our model in our scaling of the interaction coefficients.


        % Using an appropriate model of population dynamics, the raw interaction estimates can be scaled into per-capita effects (${\beta}''$) which are directly comparable between species \parencite{Godoy2014, Bimler2018}. % Note that this step depends on a population dynamic model which can link species abundance to the chosen measure of fitness, and may require species-specific measures of certain key demographic rates (e.g. mortality, seedling survival) to scale the interactions.

        % \paragraph{}
        % In order to determine the appropriate scaling for the interaction estimates returned by our framework, we transform the model into a form equivalent to a Lotka-Volterra competition model: 

        % \begin{equation}
        % F_{i} = \beta_{i0} \left ( 1 - \sum_{i}^{j} {\beta_{ij}}'' N_{j} \right )
        % \label{LVform}
        % \end{equation}

        % This gives us the scaled interaction strengths: 
 
        % \begin{equation}
        % {\beta_{ij}}'' = \frac{\beta_{ij}}{\beta_{i0}}
        % \label{scaling}
        % \end{equation}

        % \paragraph{}
        % For this case study, we adapted the framework described here to work with a well-supported annual plant population model with a seed bank \parencite{Bimler2018}. The model describes changes in the rate of a focal species' seeds in the seed bank from one year to the next, which can be linked to $F_i$ through species-specific demographic rates (equations for the annual plant model are further described in Chapter 4). The exact form of the rescaled interactions can thus vary depending on the specific population dynamic model applied and may include other demographic rates which reflect species-level differences in growth and mortality.



\section{Results}

% word count aim: 1500? 
% current word count = 1200

    \subsection{Implementation of the joint model framework}

    % 1 or 2 paragraphs explaining what we did (the 'method') applied to the data 
    \paragraph{}
        To estimate pairwise interactions in diverse, horizontal systems such as those occurring between plant species in ecological communities, we developed a joint modeling framework capable of estimating both realised and unrealised interactions from empirical observational data of individual element performance in the absence and presence of neighbours. Realised interactions are those which can be directly measured from the collected data, for example as two plant species are observed to co-occur, their performance is evaluated in each other's presence and the effect of the interaction can thus be directly quantified from the observed data. To do so, the performance of an element - a species, population or any chosen set of replicated individuals - is regressed against the abundance and identity of it's neighbouring elements in a neighbour-density dependant model (NDDM).  Increases or decreases in an overall elements' performance can thus be attributed to neighbouring elements/species which were observed to co-occur. [paste to methods? with edits on realised / unrealised]


        Key aspects of the resulting interaction matrix is that these interactions can be competitive or facilitative, non-symmetrical (the effect of element $i$ on $j$ does not necessarily match the effect of element $j$ on $i$) and include intraspecific effects (the effect of element $i$ on itself). Another versatile feature of this framework is that the shape of the relationship between performance and neighbour density can be easily modified to suit the system in question (e.g. linear, exponential, rectangular hyperbolic) though it should be carefully verified before implementation. 

    \paragraph{}
        Unrealised interactions on the other hand are those which cannot be directly measured from the collected data. As is common in ecological communities, rare elements may not be observed to co-occur with many neighbours, because data sampling is limited in space and time and rare species are less likely to be found. These rarer species may still, however, potentially interact with many of the other species in the system if given the opportunity to do so. Rather than assigning a 0 value to these unknown interactions, we used the previously quantified realised interactions from the NDDM to estimate those unrealised interactions. To do so, we describe a model which assumes that unrealised interactions can be simplified such that elements typically have a singular impact on and a singular response to neighbours independant of neighbour identity (response and impact model, or RIM). An unrealised interaction is therefore the product of two elements' impact and response parameters. Impact and response are calculated from the realised interactions provided by the NDDM and their estimation contributes directly to the overall joint model log-likelihood. Note that the NDDM estimates a unique parameter for each realised interaction independantly of each elements's effects on other neighbours, whilst the RIM applies only to those unrealised interactions. 


    \subsubsection*{Parameter estimation and validation}

        \paragraph{}
        In order to allow a seamless integration between the NDDM and RIM and for both models to contribute to the likelihood, the joint model framework was written in the STAN programming language which uses Bayesian estimation methods. This approach is also more robust to the inclusion of many interactions without running as much risk of overparameterisation, and returns parameter estimates as distributions rather than point estimates, allowing for easy inclusion of uncertainty in the analysis of results. Model parameters are sampled 1000 times from the 80\% posterior confidence intervals to construct parameter estimates. We then apply bootstrap sampling to each posterior interaction strength distribution to create 1000 samples of the community interaction matrix. 



    \subsubsection*{Application to case study data}

        The joint model was first applied to simulated data to verify that the original interaction parameters could be recovered by this framework. We then applied this model to a case study dataset of a diverse annual wildflower community, where elements consisted of 22 focal species and up to 52 neighbouring species (including focals). Seed production was used as a measure of performance and modeled with a negative binomial distribution and a log link. We conducted a posterior predictive check comparing simulated performance data to observed values (Figure \ref{fig:ppcheck}), this is especially important for verifying that the appropriate distribution and link function is being used for the data at hand. The model returned estimates for all 1144 interactions between those 22 focal species and 52 neighbouring species, of which 60\% were observed (realised).  When accounting for focal species only, 86.4\% of interactions were observed and estimated by the NDDM. NDDM and RIM estimates matched well and followed similar distributions (S.I. Figure \ref{fig:adist}) as enforced by the framework. 
    



    % \paragraph{}
    %     Jacopo made a few comments which I could also address here: 
    %     \begin{itemize}
    %         \item explore the values of r and e, e.g. whether they are correlated across species (they aren't) - can be included in Supps
    %         \item interaction estimates for the REM peak around 0 - which can only occur if $r_i$ or $e_j$ are 0 - can discuss when this might occur (1. there exists true 0's such that some species have no effects on others - 2. some species might have both positive and negative effects so averages to 0 - 3. prior is centered on 0 and signal is too weak to be picked up)


    \begin{figure}[H]
       % \hspace*{-3.5cm}
        \includegraphics[width=.6\textwidth]{../2.analyses/figures_mss/postpredch.png}
        \caption{Posterior predictive check showing the density distribution of observed seed production values (red line) to simulated seed production values (light grey), on a log scale. Simulated values were generated using the 80\% posterior confidence intervals for each parameter, the black line shows simulated values using the median of each parameter. }
        \label{fig:ppcheck}
    \end{figure}
  

    \subsection{Scaling the interactions}
    
    \paragraph{}
	The interaction matrix resulting from the joint model captures the likely effects of neighbours on the performance of focal elements. Differences in the magnitude of these interaction terms reflect differences in intrinsic performance (performance in the absence of neighbours, either competitive or facilitative) of the focal elements. In order to make interactions comparable between elements which may have very different baseline values of performance, interaction terms can be rescaled by dividing them by the returned values for intrinsic performance. 


    \paragraph{}
    To understand how the returned interaction matrix may affect system-wide behaviour, it is useful to also integrate interaction estimates into an overarching model describing the dynamics of the elements in the community. In the wildflower case study illustrated here, models of plant population dynamics are necessary to link interaction effects on performance to patterns of abundance and diversity. In our example, seed survival and germination rates were used to rescale interaction effects on seed production (our measure of performance) into \textit{per-capita} interaction effects on species growth rates. In other cases where different measures of performance are used (e.g. biomass or height), the species-specific demographic rates to be included and the rescaling formula must be derived from a system-appropriate population dynamic model, as illustrated in the Methods. The resulting per-capita interaction strengths allow us to draw inferrences about how elements may affect each other's growth and future abundances, which is particularly useful for a range of ecological applications as we further illustrate below. \\



    \begin{figure}[H]
        \begin{centering}
        % Looks like GITE is in larger font???
        \includegraphics[width=0.5\textwidth]{../2.analyses/figures_mss/networks_C_F_cooc.png}
        \caption{Competitive (A) and facilitative (B) per-capita interaction networks estimated from our model framework, compared to an association network (C) estimated from the same data using the cooccur package in R \parencite{Griffith2016}. Competitive and facilitative interactions are here shown separately for ease of view but were analysed together. Focal species only are included, arrows point to species $i$ and line thickness denotes interaction strength. Interaction strengths for (A) and (B) are given as the median over 1000 samples. Species associations (C) were all negative. 
        Purple coloured nodes correspond to highly abundant native species, whereas green nodes indicate potential keystone species, as further described in Figure \ref{fig:species}. \\
        \textbf{GITE is in larger font???}}
        \label{fig:netwks}
       \end{centering}
    \end{figure}    


    \subsection{Case study results}

    \paragraph{}
     For our case study, the resulting interaction matrix was non-symmetrical and included both positive (competitive) and negative (facilitative) values (Figure \ref{fig:netwks}. A \& B). Though competition was the dominant interaction type,  35\% of all focal x neighbour rescaled interactions were facilitative, as were 26 \% of all focal x focal interactions (averages across all network samples). As a result, 40\% of interactions between pairs of focal species were of opposing signs such that $i$ competes with $j$ but $j$ facilitates $i$. 
     Furthermore, the elements of the diagonal (the effect of an element on itself) were able to be estimated, which can allow us to quantify how much a species regulates it's own performance. For 12 of our 22 focal species, the scaled distributions of these intraspecific effects did not overlap with 0, which suggests individuals of those species have a non-trivial effect on other individuals of the same species. Together, these features make the resulting interaction matrix qualitatively different to association networks, which are a common alternative to inferring interactions in diverse systems (Figure \ref{fig:netwks}. C). 


    \paragraph{}
    We now illustrate potential uses of this framework by examining ecological questions important to species management and how the interaction matrix derived from our case study data may help us better understand the system in question. 
    We use three ecological questions to illustrate some of the ways in which our results may be useful to empirical and applied ecologists.
 
    \subsubsection*{Do abundant natives under-regulate their population density compared to rarer natives?}
    One hypothesis as to why certain plant species are more abundant than others is that they tend to compete with themselves less strongly than rare species \parencite{Yenni2012, Yenni2017}. Hypothetically, this release of intraspecific competition pressure allows them to reach much higher abundances than species which strongly compete with themselves. In Figure \ref{fig:species} A., we plot the strength of per-capita intraspecific interactions against the log abundance of a species in the field. Intraspecific interactions are at their weakest when close to $0$. The two most abundant native species \textit{Velleia rosea} (VERO) and \textit{Podolepsis canescens} (POCA) highlighted in purple fall very close to the median intraspecific interaction strength. This suggests that \textit{V. rosea} and \textit{P. canescens} do not reach high abundances through an under-regulation of their population density but by through other means (e.g. access to a larger environmental niche space). 


    \subsubsection*{Which species may be taking on keystone roles in the system?}
    Keystone species have strong effects on the dynamics of the whole ecosystem, such that their exclusion from a community can create significant changes in species density and composition \parencite{Paine1969}. Furthermore, the impact of keystone species is disproportionately large relative to their abundance \parencite{Power1996, Piraino2002, Libralato2006}. The keystone species concept is  relevant to ecosystem management and conservation in helping identify species of particular importance for the safeguarding of a whole system \parencite{Soule2005a}. Though determining which species truly serve keystone roles can be a lengthy process, we can identify potential candidates by comparing a species impact on the population growth of other species to it's own abundance \parencite{Libralato2006}. Figure \ref{fig:species} B. highlights three native species in green which may be potential keystone species due to having strongly competitive or facilitative effects on the rest of the community overall, despite rather low abundances: \textit{Trachymene ornata} (TROR), \textit{Haloragis odontocarpa} (HAOD) and \textit{Gilberta tenuifolia} (GITE). 


    \subsubsection*{Though exotic species are typically thought to compete with natives, there are also reports of facilitative exotics. Is this the case in our system?}
    Though many invasive species compete with natives \parencite{Naeem2000, Corbin2004, Riley2008, Zheng2015}, several studies have found evidence of invasives facilitating natives, with cascading effects on other species and net positive effects on ecosystem processes \parencite{Rodriguez2006, Ramus2017}. By defining per-capita interaction strengths between species in a system, we can determine which exotics are harmful or beneficial to natives. Figure \ref{fig:species} C. plots the sum of a species competitive effects on neighbours against the sum of it's facilitative effects on neighbours. Exotic species are identified in red. \textit{Hypochaeris glabra} (HYPO) and \textit{Arctotheca calendula} (ARCA) both have a lower-than-median facilitative effect on neighbours but a higher-than-median competitive effect, these two species overall have a competitive effect on the native-dominated community. \textit{Pentameris aroides} (PEAI) on the other hand has a slightly lower-than-median competitive effect and higher-than-median facilitative effect: it competes more weakly and facilitates more strongly than the median species in the system. In this instance, \textit{P. aroides} is an exotic species which nevertheless facilitates some native species.



    % \paragraph{}
    % Potentially a short paragraph on 'network' results (weighted connectance, transitivity) so that I can later discuss how the scaling + network theory can allow us to link network patterns to diversity / stability etc. Basically scaling into per-capita is necessary if we want to make inferrences about diversity maintenance or stability. \\
    % We calculated .... 



    % \begin{figure}[H]
    %    % \hspace*{-3.5cm}
    %     \includegraphics[width=\textwidth]{../../2020_Thesis/reviewer_comments/figures_for_oral_exam/intras_abundance_ch3.png}
    %     \caption{POTENTIAL FIGURE - Relationship between species abundance and how much it regulates itself (strength of per-capita intraspecific interactions). }
    %     \label{fig:aii_abund}
    % \end{figure} 

    \begin{figure}[H]
        \begin{centering}
        \includegraphics[width=0.4\textwidth]{../2.analyses/figures_mss/species_effects.png}
        \caption{(Caption next page.)}
        \label{fig:species}
        \end{centering}
    \end{figure} 

    \addtocounter{figure}{-1}
	\begin{figure} [t!]
  		\caption{(Previous page.) Understanding interaction effects can help identify species of particular ecological importance to a system. For all graphs, diamonds are species medians across all network samples, black lines cover the 50\% quantile and grey dots indicate the full range of out-strength values as calculated from 1000 sampled networks. Dashed lines represent the median value for focals. Coloured triangles indicate the species refered to in the main text for each of the ecological questions associated with (A), (B) and (C). \\
        In (A), the x-axis shows the strength of intraspecific interactions, that is how strongly a focal species interacts with itself, plotted against a focal species' log abundance (y-axis). Values over $0$ indicate competition, and values below show facilitation.  The two most abundant natives, in purple, do not seem to compete with themselves any more or less strongly than the median for all species in the system (dashed line). (B) shows the sum of interaction effects of focal species on neighbours (x-axis) against its' log abundance (x-axis). On the x-axis, values over $0$ indicate that a focal species has an overall competitive effect on neighbours, and values below $0$ indicate that is has an overall facilitative effect. Green diamonds identify species with low abundance but strong competitive or facilitative effects on neighbours. (C) decomposes a focal species' net interaction effects into its' competitive effects (x-axis) and facilitative effects. Red diamonds show exotic species.}%missing
	\end{figure}

 %    Identifying keystone species. .. GITE and HAOD rel. abund < 0.01, two highest sumaji. Maybe GOBE  too? 
 %    Invasives...
 %     Foundation species: VERO, maybe POCA ? Also well connected! 
 %    PLDE : fairly common, strong facilitative effects
 %    Dominant species are locally abundant but replaceable. 
 %    Invasives: ARCA
 %    TRCY and TROR - very different strategies!
 %    EROD: outcompeted by many species, weak effects on others, received lots of facilitation
 % - abundance - sum of comp effects - sum of facil effects



\section{Discussion}
    
% word count aim: 500-1000

% Cut three paragraphs and tighten


    \paragraph{} 
    Our novel framework quantifies the effects of interacting elements on each other's performance, allowing the estimation of diverse, horizontal interaction matrices. The resulting matrices are non-symmetrical and can contain both positive and negative interactions, as well as the effect of one element on itself. This framework is flexible to multiple metrics of performance, element identity (groups) and diversity, and provides a way to estimate un-observed interactions from those which are observed. Moreover, these matrices can be transformed into interaction networks through the use of further models describing the system's dynamics. These features make it particularly useful in an ecological context, as illustrated in our case study on a diverse wildflower community, as well as flexible to data from the wide range of complex systems which are dominated by horizontal interactions.

    \paragraph{}
    Here, we illustrate the unique useful features of this framework in an ecological context. In our case study, we show how wildflower species are linked through the resulting plant-plant interaction network. In turn, this network can help us identify what roles specific species play within a community, and explore how the mechanisms maintaining diversity and stability operate in these systems. By estimating the effects of species on each other's performance, and subsequently their population growth and patterns of abundance, this method stands in contrast to association networks. These are a common alternative to estimating interaction networks in high-diversity systems which capture spatial associations between species (e.g. \cite{Saiz2011}) and thus require easier-to-obtain data. Association networks also benefit from easy implementation with a wide range of packages in R (e.g. \cite{Griffith2016}) though the resulting networks are typically symmetrical and cannot capture a species' effect on itself. Moreover, association networks remain poor predictors of species interactions and rarely match empirical estimates \parencite{Sander2017,Barner2018, Thurman2019, Blanchet2020}.

    \paragraph{}
    Species interaction networks have a wide range of practical applications, such as evaluating ecosystem response to human-altered landscapes, guiding future management decisions \parencite{Ross2011} or exploring how communities may respond to global warming \parencite{Gorman2019}. Conservation and ecosystem management efforts aimed at regulating species abundances can, for example, use the information provided by an interaction network to prioritise which species to conserve or eradicate based on their role in the community. Such roles can be deduced by a species' position in the interaction network \parencite{Cirtwill2018a} and as illustrated in our case study. Identifying keystone, foundation and other important types of species roles is also helpful for understanding biological diversity, ecosystem integrity and functioning, especially in response to disturbances and other stresses \parencite{Nyakatya2008, Orwin2016, Losapio2017, Narwani2019}. The examples we describe in our case study are not exhaustive, but serve to illustrate how interaction networks can help us understand both community dynamics overall and the effects \& response of specific species towards the community. 

    \paragraph{}
    Quantifying the matrix of per-capita interaction strengths between species in horizontal communities can also allow us to explore how the mechanisms maintaining diversity and stability operate in these systems and across a broad number of species. Self-regulation, for example, is an extremely important driver of community stability \parencite{Barabas2017} and arises from how individuals of the same species interact with one another. Whereas same-species effects cannot be estimated by association networks, our framework quantifies intraspecific interactions and allows us to measure the strength and prevalence of competitive and facilitative density-dependence. Measures of intra and interspecific interactions can also allow us to estimate niche overlap between species (for an example, see \cite{Chu2015a}): weak interactions between species suggests they are not sharing or competing for many resources, and occupy different niche spaces in the community. %Strong competitive self-regulation, and weak interactions between two species, are two of many mechanisms leading to stability and diversity in plant communities.

    % Recent work suggests rare species are the prime emitters and beneficiaries of facilitative interactions, forming mutualistic refuges from competitive dominants \parencite{Calatayud2019, Hines2020}. Identifying neighbouring species which facilitate and increase the local abundance of rarer, endangered target species is...

    %As another example, foundation species also control local biodiversity but are widely connected, locally abundant and hold a unique position in regards to their effects on community dynamics \parencite{Ellison2005, Baiser2013, Ellison2019}. Though it can take many years of careful study to identify foundation species from abundant species which are not as crucial to community structure, our framework can help distinguish species which are abundant and interact weakly with neighbours from common species which have strong competitive and facilitative effects on the community. 
    %     % [The two species were \textit{Gilberta tenuifolia} and \textit{Haloragis odontocarpa} in case that's interesting.]


    %         \paragraph{} % MOVED FROM RESULTS
    % In addition to explicitely allowing for many species interactions, including both competitive (Figure \ref{fig:netwks}.A) and facilitative (Figure \ref{fig:netwks}.B) interactions is another feature which distinguishes our framework from traditional approaches to plant community ecology. Plant population dynamic models typically force interactions to be competitive, despite rising evidence that facilitation is common and an important driver of plant community dynamics \parencite{Brooker2008a}.

        \paragraph{}
        Another key feature of our model framework is the inclusion of facilitative interactions, which have long been disregarded in plant population models and theoretical frameworks of plant diversity-maintenance. The importance of facilitative interactions to community structure and patterns of abundance has long been recognised \parencite{Callaway1997a} but there is still little consensus on how they may affect biodiversity \parencite{Bruno2003, Brooker2008a}. Recent work suggests facilitation may be more widespread than expected \parencite{Gross2015, Picoche2020} and can  benefit species diversity and stability depending on the circumstances \parencite{Coyte2015, Brooker2008}. Our framework provides a means to investigate the prevalence and strength of facilitation across multiple species, and how it may act in relation to competition and species diversity.  
        
        \paragraph{}
        Ultimately, quantifying plant interaction networks allows us to apply tools from network theory which will help us understand not only how plants interact, but also how these interactions drive community-level patterns of abundance and diversity. Several metrics already exist for describing network structure such as weighted connectance \parencite{Ulanowicz1991} or relative intransitivity \parencite{Laird2006a}, though these are fewer than for trophic or unweighted networks networks (e.g. \cite{Bersier2002, Delmas2019}). Adapting measures of nestedness or modularity for example to non-sparse networks (as plant communities typically are) would allow us to further characterise how interactions and species are organised. These metrics relate to various aspects of stability and could greatly inform us on how diversity is maintained between plants. Likewise, networks also provide several ways of measuring and describing species roles in their respective communities \parencite{Cirtwill2018a} for example through the use of structural motifs, unique patterns of interacting species which together make up the whole network. Motifs have been found to have important biological meaning in food webs \parencite{Bascompte2005a} but remain to be identified for single-trophic and bipartite networks. 

     
    % \subsection{Future research directions}

    % - adding trait or phylogeny data, for example by informing priors with trait or phylogeny distance matrix
    % ie. improving our estimates \\
    % - improving our knowledge of networks, developing tools for plant networks and tools that can deal with non-sparse networks, competitive and facilitative interactions \\
    % - integrate with trophic networks (which appear to be sensitive to producer/plant network structure!)\\
   \subsection{Limitations}
        
        \paragraph{}
        Our novel framework provides a versatile approach to estimating interactions between system elements which are not observed to cooccur. Care should be taken, however, that these unrealised interactions do not dominate resulting networks. The method we propose uses observed interactions to estimate unobserved interactions. This method thus works better the more observed interactions available and is thus not ideal for small or heavily fragmented datasets. It is also important to consider the likely reasons for missing interactions. Forbidden links are a subset of potential interactions which cannot be observed, often due to physical constraints (e.g. biological mismatch) or spatio-temporal uncoupling. For example, a pair of short-lived annual plants might have such opposing phenologies that their growing season never overlaps in the field. This type of interaction would be assigned a 0 value. We direct the reader towards the literature on forbidden interactions \parencite{Olesen2011, Jordano2016} for solving these cases. 

        \paragraph{}
        Our model framework is aimed to helping empiricists who would like to estimate species interactions in non-trophic communities. Though the MCMC sampling algorithm does allow for many parameters to be estimated, it is crucial to check chain-mixing and model behaviour to verify that the full parameter space is sampled. The amount of data required for sampling diverse communities may still be substantial as we do not recommend fewer than 20 observations per focal species. Grouping species which appear very rarely is also a common strategy to avoid over-parameterisation. In our case study, neighbour species which were recorded fewer than 10 times across all observations were grouped into an 'other' category with its own interaction effect. If interactions with or between very rare species are the explicit object of a study, however, data-collection should focus on amassing observations of focal individuals from those rarer species to be able to estimate these interaction strengths.   

\subsection{Conclusion}

    \paragraph{} 
    There is now a rich body of work describing the characteristics of food web, plant-pollinator and host-parasite interactions, but fewer network approaches focus on non-trophic interactions such as those occurring between plants \parencite{Ellison2019}. Here we present a novel framework which makes the process of inferring  interactions in horizontal systems easier, whilst allowing for many and multiple types (competitive and facilitative) of interactions. In turn, this allows the application of  network theory tools to the management of non-trophic systems, as well as to deepening our understanding of diversity, stability and other community-level properties which emerge from interactions. Though we illustrate our study with an ecological dataset, the method presented here could be adapted for use on a wider array of  horizontal systems such as those found in microbial, neural, and social networks. 

%\paragraph{} 
 %Another approach to understanding diversity-maintenance comes from network theory, which represents an ecological community as a network of nodes or species with their interactions forming the links. This framework can include many species as well as multiple interaction types, and can be summarised by a suite of properties which describe how these species and interactions are organised. 
 % Because network theory can be used to characterise diversity, stability and other community-level properties emerging from species interactions, it has had a long and meaningful impact on our understanding of ecological communities. That said, this framework predominantly focuses on trophic webs (food web) and plant-pollinator or parasite-host systems (e.g. \cite{Lafferty2008, Thompson2012, Dunne2013, Stouffer2014, Cirtwill2015a}), with fewer methods developed for analysing the properties of horizontal communities like those of annual plant communities. 
 % Using network theory to understanding diversity-maintenance in these plant systems nonetheless presents a challenging but exciting opportunity to integrate multi-trophic and horizontal approaches to understanding the reality of species interactions and their dynamic effects on biodiversity and stability. 

    

% \bibliography{../../../BibTex_files/03_Chapter3}
% \bibliographystyle{plainnat}

\newpage

\printbibliography   

\newpage 

\section{Supplementary Methods}

    \subsection{STAN model code}

    The following code specifies the joint model framework used to estimate the interaction matrix described in the Methods. It can be copied into a text editor and saved with a .stan extension for use with STAN. 

    \includepdf[pages=-]{../../2018_Compnet/stormland/model_stan.pdf}


% \section{Supplementary Results}


    \begin{figure}[H]
       % \hspace*{-3.5cm}
        \includegraphics[width=\textwidth]{../2.analyses/figures_mss/interaction_estimates.png}
        \caption{Distribution of interaction estimates from our case study. Parameter estimates are sampled from the 80\% posterior confidence intervals returned by STAN. Upper left panel shows the distribution of observed interactions as estimated by the NDDM ($\beta_{ij}$), which are then plotted against the corresponding RIM estimates ($r_i e_j$, x-axis) in the upper right panel. Bottom rows show the distribution of \textit{unrealised} (left) and observed interactions estimates returned by the RIM. Interaction estimates are unscaled.}
        \label{fig:adist}
    \end{figure}

    %     \begin{figure}[H]
    %    % \hspace*{-3.5cm}
    %     \includegraphics[width=\textwidth]{../../2020_Thesis/reviewer_comments/figures_for_oral_exam/correlation_re.png}
    %     \caption{POTENTIAL FIGURE - Correlation between species estimates of response and impact parameters. Black triangles are species means across all samples. Note that response parameter values are constrained to be positive - this serves to prevent ... WORDING.}
    %     \label{fig:corrRE}
    % \end{figure} 


\end{document}