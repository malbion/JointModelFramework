% DOCUMENT CLASS
    % Change "letterpaper" to "a4" if you use a4 paper size
    \documentclass[a4,12pt]{article}

    \usepackage{titlesec} % Allows customization of titles
    \usepackage{authblk} % For multiple authors
    \usepackage{amsmath}
    \usepackage[utf8]{inputenc}
	\usepackage{setspace} % See \doublespacing command at the top of content.tex
    \usepackage{lineno} 	% See \linenumbers at the top of content.tex
    \usepackage{booktabs}
    \usepackage[skip=2.5\baselineskip]{caption} % to give more space between figs and captions
    \usepackage{listings} % allows me to insert code 
    \usepackage{natbib}

% GRAPHICS
    \usepackage{graphicx} % More advanced figure inclusion
    \usepackage{float} % For specifying table/figure locations, i.e. [ht!]
    \usepackage{changepage}
    \usepackage[table,xcdraw]{xcolor}

\title{Estimating interaction networks for diverse, single-trophic communities}


\author[1]{Malyon D. Bimler}
\author[2]{Daniel B. Stouffer}
\author[4]{Margaret M. Mayfield}

\affil[1]{First affiliation address, with corresponding author email. Email: malyonbimler@gmail.com}
\affil[2]{Second affiliation address}
\affil[3]{Third affiliation address}

%\setcounter{page}{54}

\begin{document}
\maketitle  
\newpage
% \linenumbers

\tableofcontents

\section{Abstract}
    
    \begin{enumerate}
    \item Understanding ecological communities requires an understanding of how the different species within it relate to each other. Those relationships are difficult to quantify in single trophic systems such as plant communities because they cannot be directly observed as in food webs or other interaction networks, and must instead be inferred by other means. Current methods are inadequate for natural complex communities because they tend to rely on co-occurrence patterns (inaccurate) or intensive experimental designs (time-consuming, especially for many species). 
    \item We describe a general framework which allows us to estimate all pair-wise interactions in complex plant communities from empirical data and population dynamics models. This framework requires a multi-tiered approach to resolve all interactions. We first apply individual fitness models to each species in order to estimate interactions between species commonly observed to co-occur. We then use the resulting interactions to approximate interactions which were not observed using a different set of assumptions. 

    %Interactions between species which rarely co-occurred are then categorised as forbidden (true 0) or unrealised. Forbidden interactions are declared on the basis of spatial, phenological or physiological mismatches. Unrealised interactions are approximated using a response-effect model.
    \item This approach can be modelled using a Bayesian statistical framework which allows us to estimate interactions coefficients despite high model complexity, as well as quantify the uncertainty around the resulting community matrix. We discuss how the flexibility afforded by the Bayesian framework may also allow us to improve interaction estimates through the inclusion of environmental, phylogenetic and funtional trait data. 
    \item We suggest this approach be used for....
    \end{enumerate}


\section{Introduction}
    
    1. Networks are important tools for understanding ecosystems: functioning, stability, diversity + roles of species. Non-trophic interactions however are often omitted from networks! 


    2. We need networks for plants but unlie common 'network' ecosystems, we can't observe interactions -> they have to be deduced.
    
    3. A common way to do this is with species association networks because they require 'simple' data, can include lots of species, and there's lots of published methods out there (papers and packages of varying complexity). Unfortunately, cooc networks don't capture interactions (even when they're super complicated?!)

    4. Another way is to estimate from growth data. This is the approach plant people take, using population dynamic models. This is often done a low diversity scales (few species) because we need a lot of replication! 
    You can use observation data but still need complex models for many interactions! Current models only estimate alphas between a few species and keep them competitive. 


    5. We describe a modelling framework + implementation which ....



    Begin with the network introduction.

    There is a growing demand for plant networks because they're 1) useful to understand ecosys functioning and 2) plants are important systems to know more about. Multiple species.    

    Building plant networks is difficult, and a major challenge to understand how plant networks influence biodiversity is characterising interactions as they can't be directly observed. A popular way around this is to build cooccurrence networks. Though cooc networks may be useful for understanding association patterns, they are not proxies for interaction networks. 'Real' interactions can be measured from growth rate / fecundity / growth data. That can be complicated when there's a lot of species involved. 
    We propose a model framework that makes this easier.
 
  
    Non trophic interactions are usually neglected in ecological networks or lumped into compartments representing other primary producers of basal resources (Baiser et al., 2013; van der Zee et al., 2016; Borst et al., 2018 - in Ellison 2019).   

    Species interaction networks are fundamentally different to co-occurrence networks as they reflect per-capita effects on species' growth rates, rather than spatial association patterns. 
    Co-occurrence networks however, benefit from being easy to implement and can naturally handle a large number of species. There are in fact many  packages available in R to calculate these networks, with methods ranging from simple (cooccur package) to extremely complex (e.g. D Harris). 
    There are few however, published plant population dynamic models which are explicitely developped to include many species.

    \begin{itemize}
        \item Kokkoris2002 =  Competition / species interactions can contribute to diversity (read more)
        \item Levine2017 - beyond pairwise mechanisms of species coexistence in complex communities (read for intro)
        \item  Montoya2006a = Ecological networks are complex (interactions too) but useful for ecological mechanisms and understanding stability. [refhole?]
        \item Kefi2015 = 'non-random patterning of non-trophic interactions suggests a path forward for developing a more comprehensive ecological network theory to rpedict the functions and resilience of ecological communities'
    \end{itemize}

    \paragraph{}
    In particular, networks for annual plant communities have been requested, in part because of how much coexistence work is done on these systems. However, building networks for plant communities is challenging: they don't interact visibly, so we need to deduce interaction strengths from growing them at different densities and in different combinations of species. Even for a species-poor community grown in a greenhouse, this is very consuming in time, money and hours. Alternatively, we can estimate interactions from field observations which record fitness e.g. seed production (refs to Godoy, Wainwright, and others who have used this method). This has the advantage of allowing us to sample very diverse communities in 'natural' conditions. However, with the problem of adding many species comes the statistical/computing difficulty of estimating this many interactions.
    \begin{itemize}
        \item Losapio2019= methods and results of plant-plant networks [refhole]
        \item Godoy2014b = field parameterised models of competitor dynamics with pairs of 18 annual plant species
    \end{itemize}


    
    \paragraph{}
    We propose a general framework to estimate interactions between plants (or other single-trophic systems) in natural and complex communities. We allow for facilitative and competitive interactions both between and within species. 
    This framework merges multiple approaches (ifm, rem) and is implemented with Bayesian stats which allows us to estimate interactions between many species. 
    We suggest avenues for applications that make use of the rich information provided by interaction networks e.g. identifying key-stone species, exploring species strategies (motifs?), looking at how invasive species affect the rest of the network, identifying species at risk of population explosion (or extinction), etc...

    % Because of how much work is required to infer interaction strengths, some people have turned to cooccurrence networks as an alternative. This means using observational data (who is abundant where) and correlation/covariance/probabilistic methods to derive positive or negative associations between species, which is argued to be a proxy for the effect of one species on another (especially if environment is taken into account?). This approach is much quicker and cheaper, and is often used to infer species interactions in species distribution models for examples (?). However, recent works suggests that co-occurrence does not acurately reflect interactions, so their usefulness in understanding coexistence between species may be limited. 

    % \begin{itemize}
    %     \item Thurman2019 = 'co-occurrence methods are generally inaccurate when estimating trophic interactions'
    %     \item Zhu2019 = co-occurrence inadequate vs DNA barcodes on gut contents (plant-herbivore)
    %     \item Sander2017 = presence-absence data doesn't allow models to consistently identify species interactions in trophic and non-trophic networks
    % \end{itemize}

    % Co-occurrence networks also suffer from not being able to measure the effect (or rather association) of a species with itself. Yet these intraspecifc interactions are key to maintaining negative density-dependence, which is a vital component of coexistence theory. It would be nice if a reference exists which also says that co-occurrence methods underestimate facilitation, which is increasingly being recognised as an important force in communities which we have yet to integrate into theoretical frameworks of coexistence (another challenge!).  
    % \citep{Verdu2010} facilitation turns to competition between plants over time. Examines the phylogenetic structure of the network. Complexity of ecological interactions and complex network approaches - examine methods?
    
    
\section{Methods}
    
    \paragraph{}
    This framework can be applied to any dataset of interacting species which meets the following criteria: 
    \begin{itemize}
        \item observations of focal individuals record a proxy for lifetime reproductive success (e.g. seed production, stem diameter growth, above-ground biomass)
        \item these observations also record the identity and abundance of neighbours within the interaction neighbourhood of each focal individual
        \item observations are replicated across several individuals of each focal species with varying neighbourhoods
    \end{itemize}
    In addition to these requirements, any experimental design or data which may reduce confounding effects between environment and competition will provide more accurate estimates of interaction strengths (e.g. thinning certain plots, recording environmental data known to affect reproductive success). 
    
    \paragraph{}   
    This framework also depends on an appropriate model of population dynamics for the system in question. The population dynamics model may require species-specific measures of certain key demographic rates (e.g. mortality, seedling survival). These rates are necessary in order to scale interaction coefficients such that they are comparable between species. 
    
    
    \subsection{Individual Fitness Model}
        
        \paragraph{}
        We begin by implementing an individual fitness model to each focal species $i$ which regresses the identity and abundance of neighbours $j$ ($j = 1, 2, 3, ...$) against the measured proxy for lifetime reproductive success $F_{i}$:
        
        \begin{equation}
        F_{i} = \beta_{i0} - \sum_{i}^{j} \beta_{ij} N_{j}
        \label{ifm}
        \end{equation}
        
        The intercept $\beta_{i0}$ represents intrinsic fitness, a species' fitness in the absence of interactions with neighbours. $N_{j}$ are the abundances of neighbours recorded for each observation, and the $\beta_{ij}$ represent the species-specific effect of each neighbour $j$ on $i$. Note that neighbours can include conspecifics, in which case intraspecific interactions are denoted as $\beta_{ii}$.
              
    \subsection{Unrealised links}
    
        \paragraph{}
        In any given site or year, a focal species may only be observed to interact with a subset of potential interaction partners, which means the IFM above will not be able to estimate all potential interactions ($\beta$) between species. This is especially true for rare species. %In order to estimate the interactions missing from the IFM, we must separate those interaction which did not occur due to chance alone (\textit{unrealised links}) from those which cannot occur due to mismatches between species (\textit{forbidden links}). 
        
        % \subsubsection{Forbidden links}
        
        % \textit{Forbidden links} are those where two species cannot interact, because they cannot co-occur or are physiologically mismatched. An example from the food web literature would be a bird which cannot eat fruit which are too big for its beak. In plants, forbidden links can occur when species are spatially or temporally mismatched, due to different environmental requirements or varying phenologies. Forbidden links may also occur between species which have vastly different resource requirements such that the presence of one species does not affect the other, though in plants these require a great detail of physiological knowledge. In this paper, we limit ourselves to estimating forbidden links from spatial mismatches. \\
        
        % \textbf{Note: I dropped the section above as very very few interactions were 'forbidden' but I can bring it back?}

        %\subsubsection{Unrealised links}
        
        %Because sampling a community is not an exhaustive process, interactions between certain species may not have been observed even though the species in question are capable of interacting. 
        These \textit{unrealised} interactions can be estimated by using an alternative model with a different set of assumptions to the IFMs. This model is described as the response-effect model (REM) by Godoy, Kraft and Levine (2014) and assumes that each species has the same effect on all neighbours regardless of their identity, as well as the same response to competition regardless of competitor identity. 
        
        \begin{equation}
        F_{i} = \beta_{i0} - \sum_{i}^{j} r_{i} e_{j} N_{j}
        \label{rem1}
        \end{equation}
        
        Pairwise interactions which are missing from Eq. 1 can be approximated by Eq. 2 by multiplying the relevant $r_{i}$ and $e_{j}$ such that $\beta_{ij} = r_{i} e_{j}$. In order to allow both the IFM and REM interaction estimates to contribute to the likelihood, we first used the IFM to quantify observed interactions and then used those to estimate species-specific $r$ and $e$ parameters such that: 
    
        \begin{equation}
        r_i e_j \sim logistic \left ( \alpha_{ij}, \sigma \right )
        \label{unrealised}
        \end{equation}
    
        where $\sigma$ is a community-level scale parameter for the logistic distribution. 
        
    \subsection{Scaling the interactions}
        
        \paragraph{}
        After applying the IFMs and estimating the remaining \textit{unrealised} interactions, the interactions effects returned by Eq. 1 and 2 must be scaled with appropriate demographic parameters determined by the system-specific model of population dynamics. This transformation puts interactions on the same scale, returning per-capita effects which are directly comparable between species (${\beta}''$) \citep{Godoy2014, Bimler2018a}. 
        
        In order to determine the appropriate scaling, the model must first be transformed into a form equivalent to a Lotka-Volterra competition model: 

        \begin{equation}
        F_{i} = \beta_{i0} \left ( 1 - \sum_{i}^{j} {\beta_{ij}}'' N_{j} \right )
        \label{LVform}
        \end{equation}

        This gives us the scaled interaction estimates: 
 
        \begin{equation}
        {\beta_{ij}}'' = \frac{\beta_{ij}}{\beta_{i0}}
        \label{scaling}
        \end{equation}

        The exact form of the rescaled interactions will vary depending on the population dynamic model applied and may include other demographic rates wich reflect species-level differences in growth and mortality.
        
    \subsection{Model fitting}

        \paragraph{}        
        The IFMs (Eq. 1) and the REM (Eq. 2) can be implemented as generalised linear models in STAN \citep{Carpenter2017}, a Bayesian statistical language where coefficient values will be estimated by MCMC sampling. The advantage of this approach is two-fold: the model can converge and coefficients can be estimated despite high model complexity and a large number of parameters. Using STAN requires translating the model formula into the STAN language, setting priors for parameters to be estimated, and using an indexing system to identify \textit{realised} interactions which are then fed into the REM. We provide a working example of the STAN file used to specify and set the model for our annual plant case study (see below) in S.I. XX. From this file, only the form of the population dynamic model and the priors need to be modified in order to apply it to a different model or system. 

        \paragraph{}
        STAN returns parameters as distributions which maximise the likelihood, and are conditioned by the data and priors. Priors describe the distribution of plausible values which these parameters may take. We recommend investigators experiment with setting different informed priors to both improve model convergence and verify the robustness of parameter estimates. The resulting parameters are termed posterior distributions, and samples from the posterior are drawn for analysis. 

    \subsection{Case study}

        \paragraph{}
        We applied this framework to annual wildflower community dataset from Western Australia. This system is a diverse and well-studied community of annual plants which germinate, grow, set seed and die within approximately 4 months every year. Individual fecundity data were collected in 2016, when 100 50 x 50 cm plots established in the understory of West Perenjori Reserve (29$^o$28'01.3"S 116$^o$12'21.6"E) were monitered over the length of the full field season, from July to October. The resulting dataset contains from 29 to over 1000 counts of individual plant seed production from 22 different focal species (with a median of 108 observations per species), in addition to the identity and abundances of all neighbouring individuals within the interaction neighbourhood of the focal plant. Interaction neighbourhoods varied in radius from 3 to 5 cm depending on the size of the focal species. Neighbouring species which were recorded less than 10 times overall were grouped into a 'rare' category. Half of all plots were thinned (a quarter to 60\% diversity and a quarter to 30\%) to mitigate possible confounding effects between plot location and plant density, and did not target any particular species.

        \paragraph{}
        We adapted the framework described here to work with a well-supported annual plant population model with a seed bank \citep{Bimler2018a}. Equations for the annual plant model are further described in Chapter 4. Species demographic rates (seed and germination) for 16 of our focal species were estimated from a database of field experiments carried out between 2016 and 2019 where seedbags were placed in the field to estimate germination rates, and ungerminated seeds were evaluated in the lab for survivability. The remaining species were assigned mean demographic rates from these experiments. 

        \paragraph{}
        We applied the model using R version 3.6.3, STAN and the rstan package \citep{R2020, Carpenter2017, Rstan2020}, running 4 chains of 5000 iterations each and discarding the first 1000. Models were checked for convergence and traceplots were visually inspected to verify good chain behaviour and mixing. Model parameters were sampled 1000 times from the 80\% posterior confidence intervals to construct our interaction estimates. We then applied bootstrap sampling from each resulting interaction strength distribution to create 1000 samples of the community interaction network.


\section{Results}

    \paragraph{}
    The model framework we present here quantifies pair-wise interaction strengths between species from diverse single-trophic community data. Several aspects of our model and its implementation allow us to better capture complex features of plant communities compared to co-occurrence networks as well as traditional plant population dynamic models. Like co-occurrence networks, our framework is flexible to the inclusion of many species. Unlike co-occurrence networks however, interactions in our model are estimated from species effects on each other's growth rates. This allows us to capture how interactions may directly affect species abundances, rather than spatial association patterns. Two features make this possible: the response-effect model to estimate unobserved interactions, and a Bayesian implementation to deal with many parameters. 

    \begin{figure}[H]
        \includegraphics[width=\textwidth]{../2.analyses/figures/alpha_est_distr.png}
        \caption{Distribution of interaction estimates from our case study. Top panel shows the distribution of observed interactions as estimated by the IFM. Second panel shows those same interactions, but estimated with the REM. Bottom panel shows the distribution of unobserved interactions estimated by the REM. Interaction estimates are given as unscaled \(\alpha\)'s.}
        \label{fig:adist}
    \end{figure}


    \paragraph{}
    The response-effect model allows us to deal with a small number of unobserved interactions by relaxing assumptions for those cases and estimating more general 'effect' and 'response' parameters from the pair-wise interactions that are observed. Because they are estimated from observed interactions previously quantified by the IFM, the response and effect parameters are constrained such that their product follows a similar distribution to that of the observed interactions (Figure \ref{fig:adist}). In our case study, 15\% of interactions between focal species were estimated using the REM (out of 484), 40\% when taking all neighbouring species into account (out of 1144). 


    \paragraph{}
    The Bayesian implementation means that parameters are estimated using MCMC algorithms, which amongst many benefits can estimate more parameters than with more commonly used frequentist approaches \citep{Dorazio2016}. This approach also returns parameter estimates as posterior distributions rather than simple mean and standard deviations. By having access to the full distribution of likely values for every interaction, we can draw samples of likely networks from these distributions and thus easily incorporate some measure of variation and uncertainty around interaction values. % This is expecially useful given how weak interactions are, and that estimates often overlap with 0...


    \paragraph{}
    The resulting networks are qualitatively different to association networks (Figure \ref{fig:netwks}.A-B, E). Firstly, intraspecific interactions ($\alpha_{ii}$) are estimated for all focal species and allow us to quantify how much a species regulates itself. In our case study, 85\% of all species across all network samples showed intraspecific competition. Interactions estimated on growth rates are also directed, which means they have an associated direction such as going from species $i$ to species $j$ ($\alpha_{ij}$), or alternatively from $j$ to $i$ ($\alpha_{ji}$). This interaction pair does not need to be of the same sign or strength, which makes the resulting interaction networks non-symmetrical. Interaction pairs of opposing signs were common in our case study (Figure \ref{fig:netwks}.D), making up 40\% of all pairs and many more differed in strength. 

    \paragraph{}
    In addition to explicitely allowing for many species interactions, including both competitive (Figure \ref{fig:netwks}.A) and facilitative (Figure \ref{fig:netwks}.B) interactions is another feature which distinguishes our framework from traditional approaches to plant community ecology. Plant population dynamic models typically force interactions to be competitive, despite rising evidence that facilitation is common and an important driver of plant community dynamics \citep{Brooker2008a}. In our case study, 35\% of all focal x neighbour interactions were facilitative, as were 26 \% of all focal x focal interactions.





    \begin{figure}[H]
        \hspace*{-2cm}
        \includegraphics[width=1.2\textwidth]{../2.analyses/figures/together.png}
        \caption{Competitive (A) and facilitative (B) interaction networks estimated from our model framework, compared to an association network (E) estimated from the same data using the cooccur package. Competitive and facilitative interactions are here shown separately for ease of view but were analysed together (C). Focal species only are included. Species associations (E) were all negative.  Interactions estimated for (A) and (B) are given as the mean over 1000 samples. (D) shows the proportion of interaction loops which were asymmetric (+/-), cooperative (+/+) or competitive (-/-).}
        \label{fig:netwks}
    \end{figure}    

    \paragraph{} 
    The resulting interaction networks can be used to identify species of particular importance to the healthy functioning of the community. Keystone species for example typically have low abundance but disproportionately strong effects on the rest of the community \citep{Power1996}. Foundation species on the other hand are locally abundant and have important, sometimes facilitative effects on other species population dynamics \citep{Ellison2019}. Figure \ref{fig:species}.A. identifies several contenders for keystone (orange diamonds) or foundation (blue diamonds) species status based on their community abundance and net interaction effects across all neighbours. Likewise, the integration of exotic species into native communities can also be evaluated using interaction networks. In our case study, two out of three exotic species (red diamonds in Fig. \ref{fig:species}) were extremely abundant but had substantially different effects on neighbours, with the grass \textit{Pentameris airoides} facilitating some species, whereas \textit{Arctotheca calendula} had almost exclusively competitive effects on all other neighbours (Fig. \ref{fig:species}.B.). These examples are not exhaustive, but serve to illustrate how interaction networks can help us understand what roles specific species play within a community.  [This last sentence might be better in the discussion]


 %    Identifying keystone species. .. GITE and HAOD rel. abund < 0.01, two highest sumaji. Maybe GOBE  too? 
 %    Invasives...
 %     Foundation species: VERO, maybe POCA ? Also well connected! 
 %    PLDE : fairly common, strong facilitative effects
 %    Dominant species are locally abundant but replaceable. 
 %    Invasives: ARCA
 %    TRCY and TROR - very different strategies!
 %    EROD: outcompeted by many species, weak effects on others, received lots of facilitation
 % - abundance - sum of comp effects - sum of facil effects

    \begin{figure}[H]
        \hspace*{-1cm}
        \includegraphics[width=1.2\textwidth]{../2.analyses/figures/spceffects_edited.png}
        \caption{Focal species net effects on neighbours, according to their abundance (A). On the x axis, values over 0 indicate a net competitive effect whereas values below 0 indicate a net facilitative effect on neighbouring species. Dashed lines represent the median value for focals. Diamonds are species means across all network samples, black lines cover the 50\% quantile and grey dots indicate the full range of out-strength values as calculated from 1000 sampled networks. Yellow, blue and red diamonds signify potential keystone or foundation species, and exotic species respectively. (B) Focal species effects on neighbours, split into competitive vs. facilitative. Axes refer to the absolute sum of competitive vs. facilitative interactions.}
        \label{fig:species}
    \end{figure} 


\section{Discussion}
    
    \paragraph{}
    The framework presented here allows the estimation of species-rich interaction matrices from plant community data. Given individual observations of reproductive success and the identity of the local neighbourhood, our model uses Bayesian regression to quantify the competitive or facilitative effects of one species on another's growth rate. This method extends traditional plant population ecology approaches to quantifying interactions by explicitely allowing for many species and interactions, as well as being able to include and estimate facilitation. We showcase this framework with a case study on a diverse annual wildflower system and illustrate how the resulting networks differ qualitatively from association networks, a common alternative to estimating interactions in species-rich communities. Quantifying non-trophic interaction networks allows us to better characterise the inner workings of these communities and has strong applications for ecosystem management and conservation. 

    \subsection{Limitations}
        
        \paragraph{}
        Though our framework does provide a way of estimating interactions between species which are not observed to cooccur, care should be taken that these unrealised interactions do not dominate a resulting network. The method we propose uses observed interactions to estimate simpler parameters which allow us to approximate the remaining interactions. This method thus works better the more observed interactions we have with which to estimate the unobserved ones. This method is not appropriate for datasets dominated by missing data on interactions (e.g. over 50\% unrealised interactions). It is also important to consider the likely reasons for missing interactions.  For instance, species pairs occupying two very different abiotic niches likely do not occur together due to environmental filtering that precedes and prevents any interaction between individuals of these two species. It would not be appropriate to include an interaction between such pairs in our framework. We direct the reader towards the literature on forbidden interactions (REFS) for solving these cases. 

        \paragraph{}
        Our model framework is aimed to helping empiricists who would like to estimate species interactions in non-trophic communities. Though the MCMC sampling algorithm does allow for many parameters to be estimated, it is crucial to check chain-mixing and model behaviour to verify that the full parameter space is sampled. The amount of data required for sampling diverse communities may still be substantial as we do not recommend fewer than 20 observations per focal species. Grouping species which appear very rarely is also a common strategy to avoid over-parameterisation. In our case study, neighbour species which were recorded fewer than 10 times in each habitat type were grouped into a 'rares' category with its own interaction effect. If interactions with or between rare species are the object of interest however, data-collection can simply be focused on amassing observations of focal individuals from those rarer species to be able to estimate these interaction strengths.   


    \subsection{Applications to ecosystem management}

        \paragraph{}
        Interaction networks provide a unique lens with which to examine an ecological community. Specifically, interaction networks allow us to determine how individual species affect and respond to community dynamics and how, in turn, community dynamics may be affected by the loss or removal of specific species (REF). This knowledge can be particularly important in guiding ecosystem management and conservation by encouraging  the establishment of management strategies which ...


        ......... are adapted to the community and species involved and acknowledge the interlinkedness of species and their effects on one another. 




        Certain species play important roles in the community which can be deduced from their position in the interaction network. Keystone species for example are both crucial to maintaining the organisation and diversity of their community, and have exceptionally strong effects on other species (Mills et al. 1993), often leading to secondary extinctions when they are removed from a network. In single-trophic systems, keystone species could be identified by their high in-strength ... Foundation species on the other hand ... Foundation species can be identified by high abundance, connectedness and predominantly facilitative interactions. \\

        Keystone species are identified as ... can be identified by having relatively low biomass in comparison to the structuring effects they have on food webs \citep{Libralato2006}. 
    Keystone species affect the communities of which they are part in a manner disproportionate to their abundance (Power et al., 1996). Keystone species strongly influence the abundances of other species and the ecosystem dynamic (Piraino et al., 2002). Therefore, it is important to identify keystone species, notably to maintain ecosystem integrity, and biological diversity in the face of exploitation and other stresses (Naeem and Li, 1997, Tilman, 2000).

    Foundation species: locally abundant and control associated biodiversity / ecosystem processes \citep{Baiser2013}. In addition, foundation species differ from dominant species in that the former are thought to be irreplaceable in terms of their control on population and community dynamics and ecosystem processes, whereas the latter are considered replaceable (Ellison et al. 2005b). \citep{Ellison2005}.

        Network approaches can also be used to determine how specific species are integrated within the community. We can determine whether invasive species are competing with or impacting many of the native species, or if they are disproportionally affecting a handful of species. Weak interactions between invasive and native species would indicate low resource sharing, such that the invasive species are probably exploiting 'empty' niche space rather taking over the niche space of natives. Threatened species would also benefit from this type of analysis, as we could deduce which other community members are hindering or helping their recovery. This is especially relevant for rare species, as recent work suggests they are the prime emitters and beneficiaries of facilitative interactions, forming mutualistic refuges from competitive dominants. If that is indeed the case, quantifying the interaction networks would help us identify which other species may be involved in 'propping up' the species of interest.  \\

        Overall, quantifying the community interaction network allows us to direct conservation and management efforts in ways which work with the community, rather than against it.

      
    \subsection{Understanding community diversity and stability}

        \paragraph{}
        Quantifying the interaction networks of plant communities allows us to explore how the mechanisms maintaining diversity and stability operate across a broad number of species. Self-regulation, for example, is an extremely important driver of community stability \citep{Barabas2017} and arises from how individuals of the same species interact with one another. Whereas same-species effects cannot be estimated by association networks, our framework quantifies intraspecific interactions and allows us to measure the strength and prevalence of competitive and facilitative density-dependence. Measures of intra and interspecific interactions can also allow us to estimate niche overlap between species (for an example, see \citet{Chu2015a}): weak interactions between species indicate they are not sharing or competing for many resources, and occupy different niche spaces in the community (ref??). Strong competitive self-regulation, and weak interactions between two species, are two of many mechanisms leading to stability and diversity in plant communities.

        \paragraph{}
        A key feature of our model framework is the inclusion of facilitative interactions, which have long been disregarded in plant population models and theoretical frameworks of plant diversity-maintenance. The importance of facilitative interactions to community structure and patterns of abundance has long been recognised \citep{Callaway1997a} but there is still little consensus on how they may affect biodiversity \citep{Bruno2003}. Recent work suggests facilitation may be more widespread than expected \citep{Gross2015, Picoche2020} and can  benefit species diversity and stability depending on the circumstances \citep{Coyte2015, Brooker2008}. Our framework provides a means to investigate the prevalence and strength of facilitation across multiple species, and how it may act in relation to competition and species diversity.  
        
        \paragraph{}
        Ultimately, quantifying plant interaction networks allows us to apply tools from network theory which will help us understand not only how plants interact, but also how these interactions drive community-level patterns of abundance and diversity. Several metrics already exist for describing network structure such as weighted connectance \citep{Ulanowicz1991} or relative intransitivity \citep{Laird2006a}, though these are fewer than for trophic or bipartite networks. Adapting measures of nestedness or modularity for example to non-sparse networks (as plant communities typically are) would allow us to further characterise how interactions and species are organised. These metrics relate to various aspects of stability and could greatly inform us on how diversity is maintained between plants. Likewise, networks also provide several ways of measuring and describing species roles in their respective communities \citep{Cirtwill2018a} for example through the use of structural motifs, unique patterns of interacting species which together make up the whole network. Motifs have been found to have important biological meaning in food webs \citep{Bascompte2005a} but remain to be identified for single-trophic and bipartite networks. 

        \paragraph{}
        Lastly



    % \subsection{Future research directions}

    % - adding trait or phylogeny data, for example by informing priors with trait or phylogeny distance matrix
    % ie. improving our estimates \\
    % - improving our knowledge of networks, developing tools for plant networks and tools that can deal with non-sparse networks, competitive and facilitative interactions \\
    % - integrate with trophic networks (which appear to be sensitive to producer/plant network structure!)\\


\bibliography{../../../BibTex_files/03_Chapter3}
\bibliographystyle{plainnat}
    

\section{Supplementary Information}

    \subsection{STAN model code}

\begin{adjustwidth}{-150pt}{}
    %\lstset{language=Stan}
    \begin{lstlisting}
                /* 
        code to run a joint IF and RE model on 
        all focal species at once
        */ 
          
          
          data {
            int<lower=1> S;          // number of species (rows in model matrix)
            int<lower=1> N;          // number of observations
            int<lower=0> K;          // number of neighbours (columns in model matrix)
            int<lower=0> I;          // number of interactions observed
            
            int<lower=0> species_ID[N];   // species index for observations
            int<lower=0> seeds[N];        // response variable 
            
            int<lower=0> istart[S];       // indices matching species to interactions
            int<lower=0> iend[S];
            int<lower=0> icol[I];
            int<lower=0> irow[I];
            
            matrix[N,K] X;         // neighbour abundances, the model matrix
          
          } 

        parameters {
          
          vector[S] a;    // species-specific intercept, i.e. log(lambda) (intrinsic fitness)
          vector<lower=0>[S] disp_dev; // species-specific param for dispersion deviation, 
          // disp_dev = 1/sqrt(phi) 
          
          
          vector[I] interactions;     // vector of interactions which have been observed
          vector[K] effect;            // competitive effect of neighbours, same across all 
          // focals, can be facilitative (-) or competitive (+)
          vector<lower=0>[S] response; // species-specific competitive response parameter
          // >= 0 to avoid bimodality in response and effect  

          real<lower=0> sigma_alph; // variance for the ifm alphas
        } 

        transformed parameters {
          
          // transformed parameters constructed from params above
          vector[N] mu;              // the linear predictor
          matrix[S, K] ifm_alpha;    // community matrix 
          vector[I] re;              // interactions as calculated by the re model
          
          ifm_alpha = rep_matrix(0, S, K); // fill the community matrix with 0 (instead of NA)
          
          // match observed interactions parameters to the correct position in the community matrix
          for(s in 1:S) {
            for(i in istart[s]:iend[s]) {
              ifm_alpha[irow[i], icol[i]] = interactions[i];
            }
          }
          
          // individual fitness model 
          for(n in 1:N) {
               mu[n] = exp(a[species_ID[n]] - dot_product(X[n], ifm_alpha[species_ID[n], ]));  
          }
          
          // build a vector of interaction parameters based on the response effect model 
          for (i in 1:I) {
            re[i] = response[irow[i]]*effect[icol[i]];
          }
          
        } 

        model {

          // priors
          a ~ cauchy(0,10); // prior for the intercept following Gelman 2008
          disp_dev ~ cauchy(0, 1);  // safer to place prior on disp_dev than on phi
          
          response ~ normal(0, 1);   // 
          effect ~ normal(0, 1);
          sigma_alph ~ cauchy(0, 1);

          // seed production 
          for(n in 1:N) {
            seeds[n] ~ neg_binomial_2(mu[n], (disp_dev[species_ID[n]]^2)^(-1));
          }

          // response-effect interactions
          for (i in 1:I) {
            target += logistic_lpdf(re[i] | interactions[i], sigma_alph);
          }
          
        }
    \end{lstlisting}

\end{adjustwidth}

\end{document}